% Copyright 2004 by Till Tantau <tantau@users.sourceforge.net>.
%
% In principle, this file can be redistributed and/or modified under
% the terms of the GNU Public License, version 2.
%
% However, this file is supposed to be a template to be modified
% for your own needs. For this reason, if you use this file as a
% template and not specifically distribute it as part of a another
% package/program, I grant the extra permission to freely copy and
% modify this file as you see fit and even to delete this copyright
% notice. 

\documentclass{beamer}

\usepackage{soul}

\usefonttheme{professionalfonts} % using non standard fonts for beamer
\usefonttheme{serif} % default family is serif
%\usepackage{fontspec}
%\setmainfont{Liberation Serif}

% There are many different themes available for Beamer. A comprehensive
% list with examples is given here:
% http://deic.uab.es/~iblanes/beamer_gallery/index_by_theme.html
% You can uncomment the themes below if you would like to use a different
% one:
%\usetheme{AnnArbor}
%\usetheme{Antibes}
%\usetheme{Bergen}
%\usetheme{Berkeley}
%\usetheme{Berlin}
%\usetheme{Boadilla}
%\usetheme{boxes}
%\usetheme{CambridgeUS}
%\usetheme{Copenhagen}
%\usetheme{Darmstadt}
\usetheme{default}
%\usetheme{Frankfurt}
%\usetheme{Goettingen}
%\usetheme{Hannover}
%\usetheme{Ilmenau}
%\usetheme{JuanLesPins}
%\usetheme{Luebeck}
%\usetheme{Madrid}
%\usetheme{Malmoe}
%\usetheme{Marburg}
%\usetheme{Montpellier}
%\usetheme{PaloAlto}
%\usetheme{Pittsburgh}
%\usetheme{Rochester}
%\usetheme{Singapore}
%\usetheme{Szeged}
%\usetheme{Warsaw}



\title{Digital Signal Processing: Theory and Practice}

% A subtitle is optional and this may be deleted
\subtitle{Course Introduction}

\author{Sivakumar Balasubramanian}
% - Give the names in the same order as the appear in the paper.
% - Use the \inst{?} command only if the authors have different
%   affiliation.

\institute[Christian Medical College] % (optional, but mostly needed)
{
  \inst{}%
  Department of Bioengineering\\
  Christian Medical College, Bagayam\\
  Vellore 632002
}
% - Use the \inst command only if there are several affiliations.
% - Keep it simple, no one is interested in your street address.

\date{}
% - Either use conference name or its abbreviation.
% - Not really informative to the audience, more for people (including
%   yourself) who are reading the slides online

\subject{Lecture notes on signal processing}
% This is only inserted into the PDF information catalog. Can be left
% out. 

% If you have a file called "university-logo-filename.xxx", where xxx
% is a graphic format that can be processed by latex or pdflatex,
% resp., then you can add a logo as follows:

% \pgfdeclareimage[height=0.5cm]{university-logo}{university-logo-filename}
% \logo{\pgfuseimage{university-logo}}

% Delete this, if you do not want the table of contents to pop up at
% the beginning of each subsection:
\AtBeginSubsection[]
{
  \begin{frame}<beamer>{Outline}
    \tableofcontents[currentsection,currentsubsection]
  \end{frame}
}

% Let's get started
\begin{document}

\begin{frame}
  \titlepage
\end{frame}

% WHAT IS THE COURSE ABOUT?
\begin{frame}[t]{What is the course about?}
\begin{itemize}
\item Course on the theory and practice of digital signal processing techniques.
\item Primary focus will be on understanding:
\begin{itemize}
\item Foundations of modern signal processing.
\item Discrete-time signal representation and analysis.
\item Discrete time system analysis and synthesis.
\item Design, implementation and analysis of frequency-selective filters.
\item Understanding and analysis of practical issues in real-time DSP algorithm implementation.
\item Hands on experience in applying theory to solve real problems.
\end{itemize}
\end{itemize}
\end{frame}

% WHAT TO EXPECT FROM THE COURSE?
\begin{frame}[t]{What to expect from the course?}
\begin{itemize}
\item An introduction to the foundations of signal processing.
\item A good understanding of the theory discrete-time signals and systems.
\item Ability to analyze and synthesize digital filters.
\item Ability to practically implement DSP algorithms in hardware.
\end{itemize}
\end{frame}

% PRE-REQUISITES
\begin{frame}[t]{Pre-requisites}
\begin{itemize}
\item Basic understanding of real and complex analysis.
\item Basic understanding of calculus (limits, differentiation, integration).
\item Experience in programming (C and Python (or Matlab) would be ideal).
\end{itemize}
\end{frame}

% COURSE LAYOUT
\begin{frame}[t]{Course Scoring and Grading}

\textbf{Total: 100} [20 + 20 + 15 + 45]
\begin{itemize}
\item Lab assignments: 20
\item Surprise quiz: 20
\item Mid-term: 15
\item Final: 45
\end{itemize}
\textbf{Late submissions will be corrected but not graded.}
\end{frame}

\begin{frame}[t]{Course Scoring and Grading}

\textbf{Grading policy}: \textbf{\ul{No relative grading}}
\begin{itemize}
\item A+: Score $\geq 90/100$
\item A: $80 \leq$ Score $< 90$
\item B: $70 \leq$ Score $< 80$
\item C: $60 \leq$ Score $< 70$
\item D: $50 \leq$ Score $< 60$
\item E: $40 \leq$ Score $< 50$
\item F: Score $< 40$
\end{itemize}
\textbf{Academic dishonesty will automatically fetch the person a `F' grade.}
\end{frame}

% COURSE CONTENT
\begin{frame}{Course content}
\begingroup
    \fontsize{9pt}{12pt}\selectfont
    \begin{enumerate}
    \item Mathematical preliminaries
    \item What are digital signals and systems?
    \item Some useful and important signals
    \item Sampling, quantization and number representation
    \item Geometric signal theory
    \item Discrete­time LTI systems
    \item Discrete­-time Fourier transform and its properties
    \item Z-­transform and its properties
    \item Analysis of discrete-­time LTI systems
    \item Discrete Fourier Transform and its properties
    \item Fast Fourier Transform
    \item Design and analysis of digital filters
    \item Practical aspect of DSP algorithm implementation
    \item Understanding random signals
    \item Spectral analysis
    \end{enumerate}  
\endgroup
\end{frame}

\end{document}


